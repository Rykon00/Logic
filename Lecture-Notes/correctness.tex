\chapter{Correctness Proofs}
In this chapter we will show two different methods that can be used to prove the correctness of a \textsl{Python}
function.
\begin{enumerate}[(a)]
\item The method of \blue{computational induction} can be used to verify the correctness of a \textsl{Python}
      function that is defined recursively.
\item In order to establish the correctness of a \textsl{Python} function that is defined iteratively we use
      \blue{symbolic execution}. 
\end{enumerate}

\section{Computational Induction}
Figure \ref{fig:power.py} shows the definition of the function $\mytt{power}(m,n)$ that computes
the value $m^n$.  We will verify the correctness of this function.

\begin{figure}[!h]
  \centering
\begin{minted}[ frame         = lines, 
                framesep      = 0.3cm, 
                numbers       = left,
                numbersep     = -0.2cm,
                bgcolor       = sepia,
                xleftmargin   = 0.8cm,
                xrightmargin  = 0.8cm
              ]{python3}
    def power(m, n):
        if n == 0:
            return 1
        p = power(m, n // 2)
        if n % 2 == 0:
            return p * p
        else:
            return p * p * m
\end{minted}
\vspace*{-0.3cm}
  \caption{Computation of $m^n$ for $m,n \in \mathbb{N}$.}
  \label{fig:power.py}
\end{figure} 

It is by no means obvious that the program shown in \ref{fig:power.py} does compute
$m^n$.  We prove this claim by  \blue{computational induction}\index{computational induction}.
Computational induction is an induction on the number of recursive invocations.
This method is the method of choice to prove the correctness of a function if this function is defined recursively.
A proof by computational induction consists of three parts:
\pagebreak

\begin{enumerate}
\item The \blue{base case}.

      In the base case we have to show that the function definition is correct in all those cases where the function
      does not invoke itself recursively.
\item The \blue{induction step}.

      In the induction step we have to prove that the function definition works in all those cases where
      the function does invoke itself recursively.  In order to carry out this proof we may
      assume that the results computed by the recursively invocations are correct.
      This assumption is called the \blue{induction hypotheses}.
\item The \blue{termination proof}.

      In this final step we have to show that the recursive definition of the function is \blue{well founded},
      i.e.~we have to prove that the recursive invocations terminate.
\end{enumerate}
Let us prove the claim 
\\[0.2cm]
\hspace*{1.3cm}
 $\mytt{power}(m,n) = m^n$
\\[0.2cm] 
by computational induction.
\begin{enumerate}
\item \textbf{Base case}:

      The only case where \mytt{power} does not invoke itself recursively is the case $n = 0$.  
      In this case, we have
      \\[0.2cm]
      \hspace*{1.3cm} 
      $\mytt{power}(m,0) = 1 =  m^0$. \mycheck
\item \textbf{Induction step}:

      The recursive invocation of $\mytt{power}$ has the form
      $\mytt{power}(m,n \dv 2)$.  By the induction hypotheses we may assume that 
      \\[0.2cm]
      \hspace*{1.3cm}
      $\displaystyle \mytt{power}(m,n \dv 2) = m^{n \dvv 2}$ 
      \\[0.2cm]
      holds.  After the recursive invocation there are two cases that have to be dealt with separately.
      \begin{enumerate}
      \item $n \;\mytt{\%}\; 2 = 0$, therefore $n$ is even.

            Then there exists a number $k \in \mathbb{N}$ such that $n = 2 \cdot k$ and therefore
            $n \dv 2 = k$.
            Hence we have:
            \\[0.2cm]
            \hspace*{1.3cm}
           $ 
            \begin{array}{lcl}
            \mytt{power}(m,n) & = & \mytt{power}(m,k) \cdot \mytt{power}(m,k) \\[0.2cm]
                                & \stackrel{\mathrm{IV}}{=} & m^k \cdot m^k  \\[0.2cm]
                                & = & m^{2\cdot k} \\[0.2cm]
                                & = & m^{n}.
            \end{array}
            $            
      \item $n \;\mytt{\%}\; 2 = 1$, therefore $n$ is odd.

            Then there exists a number $k \in \mathbb{N}$ such that $n = 2 \cdot k + 1$ and we have
            $n \dv 2 = k$.  In this case we have:
            \\[0.2cm]
            \hspace*{1.3cm}
            $ 
            \begin{array}{lcl}
            \mytt{power}(m,n) & = & \mytt{power}(m,k) \cdot \mytt{power}(m,k) \cdot m  \\[0.2cm]
                                & \stackrel{\mathrm{IV}}{=} & m^k \cdot m^k \cdot m  \\[0.2cm]
                                & = & m^{2\cdot k+1} \\[0.2cm]
                                & = & m^{n}.
            \end{array}
            $
      \end{enumerate}
      As we have shown that $\mytt{power}(m,n) = m^n$ in both cases, the induction step is finished. \mycheck
\item \textbf{Termination proof}:
      Every time the function \mytt{power} is invoked as $\mytt{power}(m, n)$ and $n > 0$, the recursive
      invocation has the form $\mytt{power}(m,n \dv 2)$ and, since $n \dv 2 < n$ for all $n > 0$, the second
      argument is decreased.  As this argument is a natural number, it must eventually reach $0$.  But if the
      second argument of the function $\mytt{power}$ is $0$, the function terminates immediately. \mycheck
      \qed
\end{enumerate}

\begin{figure}[!h]
  \centering
\begin{minted}[ frame         = lines, 
                framesep      = 0.3cm, 
                numbers       = left,
                numbersep     = -0.2cm,
                bgcolor       = sepia,
                xleftmargin   = 0.8cm,
                xrightmargin  = 0.8cm
                ]{python3}
    def div_mod(m, n):
        if m < n:
            return 0, m
        q, r = div_mod(m // 2, n)
        if 2 * r + m % 2 < n:
            return 2 * q    , 2 * r + m % 2
        else:
            return 2 * q + 1, 2 * r + m % 2 - n                
\end{minted}
\vspace*{-0.3cm}
  \caption{The function \mytt{div\_mod}.}
  \label{fig:div_mod}
\end{figure} 


\exampleEng
The function \mytt{div\_mod} that is shown in Figure \ref{fig:div_mod} satisfies the specification
\\[0.2cm]
\hspace*{1.3cm}
$\mytt{div\_mod}(m, n) = (q, r)  \rightarrow m = q \cdot n + r \;\wedge\; r < n$. \eox



\proof
Assume that $m,n \in \mathbb{N}$, where $n > 0$.  Furthermore, assume
\\[0.2cm]
\hspace*{1.3cm}
$\bar{q}, \bar{r} = \mytt{div\_mod}(m, n)$.
\\[0.2cm] 
In order to prove the correctness of \mytt{div\_mod}, we have to show two formulas:
\begin{align}
m = \bar{q} \cdot n + \bar{r} \label{divmod1} \\
\bar{r} < n \hspace*{1.1cm} \label{divmod2}
\end{align}
Since \mytt{div\_mod} is defined recursively, the proof of these formulas is done by computational induction.
\begin{enumerate}
\item[B.C.:] $m < n$

  In this case we have $\bar{q} = 0$ and $\bar{r} = m$.
  In order to prove (\ref{divmod1}) we note that 
  \begin{align*}
                    & m = \bar{q} \cdot n + \bar{r} \\
    \Leftrightarrow\quad & m = 0 \cdot n + m \quad \green{\surd}
  \end{align*}
  To prove (\ref{divmod2}) we note that
  \begin{align*}
                         & \bar{r} < n \\
    \Leftrightarrow\quad & m < n \quad\green{\surd}
  \end{align*}
  Here $m < n$ is true because this condition is the assumption of the base case.
\item[I.S.:] $m \dv 2 \mapsto m$

  By induction hypotheses we know that our claim is true for the recursive invocation of \mytt{div\_mod} in
  line 4.  Therefore we have the following:
  \begin{eqnarray}
    m \dv 2 &\!=\! & q \cdot n + r \label{divmod3} \\
          r &\!<\! & n             \label{divmod4}
  \end{eqnarray}
  In order to complete the induction step we have to perform a case distinction that is analogous to the test
  of the second \texttt{if}-statement in the implementation of \mytt{div\_mod}.
  \begin{enumerate}
  \item $2 \cdot r + m \mmod 2 < n$

    In this case we have $\bar{q} = 2 \cdot q$ and $\bar{r} = 2 \cdot r + m \mmod 2$.
    In order to prove (\ref{divmod1}) we note the following:
    \begin{align}
                           &  m = \bar{q} \cdot n + \bar{r} \\
      \Leftrightarrow\quad & m = 2 \cdot q \cdot n + 2 \cdot r + m \mmod 2 \label{divmod5}
    \end{align}
    We will derive equation (\ref{divmod5}) from equation (\ref{divmod3}).  To this end, we multiply equation
    (\ref{divmod3}) by $2$.  This yields:
    \\[0.2cm]
    \hspace*{1.3cm}
    $2 \cdot m \dv 2 = 2 \cdot q \cdot n + 2 \cdot r$.
    \\[0.2cm]
    If we add $m \mmod 2$ to this equation we get
    \\[0.2cm]
    \hspace*{1.3cm}
    $2 \cdot m \dv 2 + m \mmod 2 = 2 \cdot q \cdot n + 2 \cdot r + m \mmod 2$.
    \\[0.2cm]
    As we have $2 \cdot m \dv 2 + m \mmod 2 = m$ the last equation can be simplified to
    \\[0.2cm]
    \hspace*{1.3cm}
    $m = 2 \cdot q \cdot n + 2 \cdot r + m \mmod 2$.
    \\[0.2cm]
    However, this is just equation (\ref{divmod5}) which we had to prove. \mycheck

    Next, we show that $\bar{r} < n$.  This is equivalent to
    \\[0.2cm]
    \hspace*{1.3cm}
    $2 \cdot r + m \mmod 2 < n$.
    \\[0.2cm]
    However, this inequation is the condition of this case of the case distinction and is therefore valid. \mycheck
  \item $2 \cdot r + m \mmod 2 \geq n$

    In this case we have $\bar{q} = 2 \cdot q + 1$ and $\bar{r} = 2 \cdot r + m \mmod 2 - n$.
    We start with the proof of (\ref{divmod1}).
    \begin{align*}
                           & m = \bar{q} \cdot n + \bar{r} \\
      \Leftrightarrow\quad & m = (2 \cdot q + 1) \cdot n + 2 \cdot r + m \mmod 2 - n \\
      \Leftrightarrow\quad & m = 2 \cdot q \cdot n + 2 \cdot r + m \mmod 2 
    \end{align*}
    This last equation follows from equation (\ref{divmod3}) as follows:
    \begin{align*}
                       & m \dv 2 = q \cdot n + r \\
      \Rightarrow\quad & 2 \cdot m \dv 2 = 2 \cdot q \cdot n + 2 \cdot r \\
      \Rightarrow\quad & 2 \cdot m \dv 2 + m \mmod 2 = 2 \cdot q \cdot n + 2 \cdot r + m \mmod 2 \\
      \Rightarrow\quad & m = 2 \cdot q \cdot n + 2 \cdot r + m \mmod 2 
    \end{align*}

    Next, we show that $\bar{r} < n$. This is equivalent to
    \\[0.2cm]
    \hspace*{1.3cm}
    $2 \cdot r + m \mmod 2 - n < n$
    \\[0.2cm]
    From (\ref{divmod4}) we know that
    \begin{align*}
                       & r < n \\
      \Rightarrow\quad & r + 1 \leq n \\
      \Rightarrow\quad & 2 \cdot r + 2 \leq 2 \cdot n \\
      \Rightarrow\quad & 2 \cdot r + m \mmod 2 + 1 \leq 2 \cdot n \quad \mbox{since $m \mmod 2 \leq 1$} \\
      \Rightarrow\quad & 2 \cdot r + m \mmod 2 < 2 \cdot n \\
      \Rightarrow\quad & 2 \cdot r + m \mmod 2 - n < n \green{\surd}
    \end{align*}
  \end{enumerate}
\item[T.:] As $m \dv 2 < m$ for all $m \geq n$ and $n > 0$ it is obvious that we will eventually have
  $m  < n$. But then the function \mytt{div\_mod} terminates.
\end{enumerate}

Before we can tackle the next exercise, we need to prove the following lemma.
\begin{Lemma}[Euclid]
  Assume $a, b \in \mathbb{N}$ such that $b > 0$.  Then we have
  \\[0.2cm]
  \hspace*{1.3cm}
  $\mathtt{gcd}(a, b) = \mathtt{gcd}(b, a \mmod b)$.
\end{Lemma}

\proof
The function $\mathtt{cd}(a, b)$ computes the set of common divisors of $a$ and $b$ and is therefore defined as
\\[0.2cm]
\hspace*{1.3cm}
 $\mathtt{cd}(a, b) := \bigl\{ t \in \mathbb{N} \mid a \mmod t = 0 \wedge b \mmod t = 0 \bigr\}$.
 \\[0.2cm]
The function \texttt{gcd} is related to the function \texttt{cd} by the equation
\\[0.2cm]
\hspace*{1.3cm}
$\mathtt{gcd}(a, b) = \max\bigl(\mathtt{cd}(a, b)\bigr)$.
\\[0.2cm]
Hence it is sufficient if we can show that
\\[0.2cm]
\hspace*{1.3cm}
  $\mathtt{cd}(a, b) = \mathtt{cd}(b, a \mmod b)$.
\\[0.2cm]
This is an equation between two sets and therefore is equivalent to showing that both
\\[0.2cm]
\hspace*{1.3cm}
  $\mathtt{cd}(a, b) \subseteq \mathtt{cd}(b, a \mmod b)$ \quad and \quad $\mathtt{cd}(b, a \mmod b) \subseteq \mathtt{cd}(a, b)$
\\[0.2cm]
holds.  We show these two statements separately.
\begin{enumerate}
\item We show that $\mathtt{cd}(a, b) \subseteq \mathtt{cd}(b, a \mmod b)$.

      Assume $t \in \mathtt{cd}(a, b)$.  Then there are $u, v \in \mathbb{N}$ such that
      \\[0.2cm]
      \hspace*{1.3cm}
      $a = t \cdot u$ \quad and \quad $b = t \cdot v$.
      \\[0.2cm]
      Since $a = q \cdot b + a \mmod b$ where $q = a \dv b$ we have
      \\[0.2cm]
      \hspace*{1.3cm}
      $a \mmod b = a - q \cdot b = t \cdot u - q \cdot t \cdot v = t \cdot (u - q \cdot v)$.
      \\[0.2cm]
      This shows that $t$ divides $a \mmod b$. Since $t$ also divides $b$ we therefore have
      $t \in \mathtt{cd}(b, a \mmod b)$. \green{$\surd$}
\item We show that $\mathtt{cd}(b, a \mmod b) \subseteq \mathtt{cd}(a, b)$.

      Assume $t \in \mathtt{cd}(b, a \mmod b)$.  Then there are $u, v \in \mathbb{N}$ such that
      \\[0.2cm]
      \hspace*{1.3cm}
      $b = t \cdot u$ \quad and \quad $a \mmod b = t \cdot v$.
      \\[0.2cm]
      Since $a = q \cdot b + a \mmod b$ where $q = a \dv b$ we have
      \\[0.2cm]
      \hspace*{1.3cm}
      $a = q \cdot t \cdot u + t \cdot v = t \cdot (q \cdot u + v)$.
      \\[0.2cm]
      This shows that $t$ divides $a$. Since $t$ also divides $b$ we therefore have
      $t \in \mathtt{cd}(a, b)$. \green{$\surd$}
\end{enumerate}
This concludes the proof. \qed

\begin{figure}[!h]
  \centering
\begin{minted}[ frame         = lines, 
                framesep      = 0.3cm, 
                numbers       = left,
                numbersep     = -0.2cm,
                bgcolor       = sepia,
                xleftmargin   = 0.8cm,
                xrightmargin  = 0.8cm
                ]{python3}
    def ggt(x, y):
        if y == 0:
            return x
        return ggt(y, x % y)
\end{minted}
\vspace*{-0.3cm}
  \caption{The function \mytt{ggt}.}
  \label{fig:gcd}
\end{figure}

\exerciseEng
Prove that the function \mytt{ggt} that is shown in Figure \ref{fig:gcd} computes the greatest common divisor
of its arguments. \eox

\begin{figure}[!h]
  \centering
\begin{minted}[ frame         = lines, 
                framesep      = 0.3cm, 
                numbers       = left,
                numbersep     = -0.2cm,
                bgcolor       = sepia,
                xleftmargin   = 0.8cm,
                xrightmargin  = 0.8cm
                ]{python3}
    def root(n):
        if n == 0:
            return 0
        r = root(n // 4)
        if (2 * r + 1) ** 2 <= n:
            return 2 * r + 1
        else:
            return 2 * r
\end{minted}
\vspace*{-0.3cm}
  \caption{The function \mytt{isqrt}.}
  \label{fig:isqrt}
\end{figure} 

\exerciseEng
The \blue{integer square root} of a natural number $n$ is defined as 
\\[0.2cm]
\hspace*{1.3cm}
$\texttt{isqrt}(n) := \max\bigl(\{ r \in \mathbb{N} \mid r^2 \leq n \}\bigr)$.
\\[0.2cm]
Prove that the function \mytt{root} that is shown in Figure \ref{fig:isqrt} on page \pageref{fig:isqrt}
computes the integer square root of its argument. \eox
\pagebreak

\exerciseEng
Figure \ref{fig:square} shows a \textsl{Python} program implementing the function \texttt{square}.  Show that the equation
\\[0.2cm]
\hspace*{1.3cm}
$\mathtt{square}(n) = n^2$
\\[0.2cm]
holds for every $n \in \mathbb{N}$.  You should use computational induction to verify the claim. \eox

\begin{figure}[!ht]
\centering
\begin{minted}[ frame         = lines, 
                framesep      = 0.3cm, 
                numbers       = left,
                numbersep     = -0.2cm,
                bgcolor       = sepia,
                xleftmargin   = 0.8cm,
                xrightmargin  = 0.8cm,
              ]{python3}
    def square(n):
        if n == 0:
            return 0
        return square(n-1) + 2 * n - 1 
\end{minted}
\vspace*{-0.3cm}
\caption{The function \texttt{square}.}
\label{fig:square}
\end{figure}

\exerciseEng
The \href{https://en.wikipedia.org/wiki/Binomial_coefficient}{binomial coefficient} $\ds {n \choose k}$ is defined as follows:
\\[0.2cm]
\hspace*{1.3cm}
$\ds {n \choose k} = \frac{n!}{k! \cdot (n-k)!}$
\\[0.2cm]
Figure \ref{fig:binomial} shows a \textsl{Python} program implementing the function \texttt{binomial}.  Prove
that 
\\[0.2cm]
\hspace*{1.3cm}
$\ds \mathtt{binomial}(n, k) = {n \choose k}$
\\[0.2cm]
holds for all $n \in \mathbb{N}$ and all $k \in \{ 0, \cdots, n \}$. \eox

\begin{figure}[!ht]
\centering
\begin{minted}[ frame         = lines, 
                framesep      = 0.3cm, 
                numbers       = left,
                numbersep     = -0.2cm,
                bgcolor       = sepia,
                xleftmargin   = 0.8cm,
                xrightmargin  = 0.8cm,
                ]{python3}
    def binomial(n, k):
        if k == 0 or k == n:
            return 1
        else:
            return binomial(n - 1, k - 1) + binomial(n - 1, k)                
\end{minted}
\vspace*{-0.3cm}
\caption{The function \texttt{binomial}.}
\label{fig:binomial}
\end{figure}



\section{Symbolic Execution}
In the last chapter we have seen how to prove the correctness of a recursive function via
\blue{computational induction}.  If a function is implemented iteratively via loops instead of recursively,
then the method of computational induction is not applicable.   Instead, the method of 
\blue{symbolic execution} \index{symbolic execution} is then used.
We introduce this method via two examples.

\subsection{Iterative Squaring}
In the previous section we have implemented a recursive version of \blue{iterative squaring} and verified its
correctness via \blue{computational induction}.  In this section we implement an iterative version of
\blue{iterative squaring} and verify its correctness via \blue{symbolic execution}.
Consider the program shown in Figure \ref{fig:power-iterative-annotated.stlx}. 

\begin{figure}[!h]
\centering
\begin{Verbatim}[ frame         = lines, 
                  framesep      = 0.3cm, 
                  labelposition = bottomline,
                  numbers       = left,
                  numbersep     = -0.2cm,
                  xleftmargin   = 1.3cm,
                  xrightmargin  = 1.3cm,
                  codes         = {\catcode`_=8\catcode`$=3},
                  commandchars  = \\\{\},
                ]
    \colorbox{sepia}{\green{def} \blue{power}(x$_0$, y$_0$):}
    \colorbox{sepia}{    r$_0$ = 1}
    \colorbox{sepia}{    \green{while} y$_n$ > 0:}
    \colorbox{sepia}{        \green{if} y$_n$ % 2 == 1:}
    \colorbox{sepia}{            r$_{n+1}$ = r$_n$ * x$_n$}
    \colorbox{sepia}{        x$_{n+1}$ = x$_n$ * x$_n$}
    \colorbox{sepia}{        y$_{n+1}$ = y$_n$ // 2}
    \colorbox{sepia}{    \green{return} r$_{N}$}
\end{Verbatim}
\vspace*{-0.3cm}
\caption{An annotated programm to compute powers.}
\label{fig:power-iterative-annotated.stlx}
\end{figure} % $

The main difference between a mathematical formula and a program is that in a formula all
occurrences of a variable refer to the same value.   This is different in a program because the
variables change their values dynamically.  In order to deal with this property of program variables,
we have to be able to distinguish the different occurrences of a given variable.  To this end,  we 
\blue{index} the program variables. 
When doing this, we have to be aware of the fact that the same occurrence of a program variable can
still denote different values if the variable occurs inside  a loop.  In this case we have to index
the variables in a way such that the index includes a counter that counts the number of loop iterations.
For concreteness, consider the  program shown in 
Figure \ref{fig:power-iterative-annotated.stlx}.  
Here, in line 5 the variable \mytt{r} has the index $n$ on the right side of the assignment,
while it has the index $n+1$ on the left side of the assignment in line 5.  The index $n$ denotes 
the number of times that the test \mytt{y > 0} of the \mytt{while} loop has been executed.
After the \texttt{while}-loop finishes, the variable $\mytt{r}$ is indexed as
$\mytt{r}_N$ in line 8, where $N$ denotes the total number of times that the test \mytt{y > 0} has been executed.
We show the correctness of the given program next.  Let us define
\\[0.2cm]
\hspace*{1.3cm}
$ a := x_0, \quad b := y_0$.
\\[0.2cm]
We will show that, provided $a \geq 1$, the \mytt{while} loop satisfies the \blue{invariant}
\begin{equation}
  \label{eq:powerInv}
  \ds r_n \cdot x_n^{y_n} = a^b.
\end{equation}
This claim is proven by induction on the number of loop iterations.
\begin{enumerate}
\item[B.C.:] $n=0$.

            Since we have $r_0 = 1$, $x_0 = a$, and $y_0 = b$ we have 
            \\[0.2cm]
            \hspace*{1.3cm}
            $r_n \cdot x_n^{y_n} = r_0 \cdot x_0^{y_0} = 1 \cdot a^{b} = a^b$. \mycheck
\item[I.S.:] $n \mapsto n + 1$.

            We proof proceeds by a case distinction with respect to the expression $y_n \mmod 2$:
            \begin{enumerate}
            \item $y_n \mmod 2 = 1$.

                  Then we have $y_{n} = 2 \cdot (y_n\dv 2) + 1$ and
                  $r_{n+1} = r_n \cdot x_n$.  Hence
                  \begin{eqnarray*}
                      &   & r_{n+1} \cdot x_{n+1}^{y_{n+1}} \\[0.2cm] 
                      & = & (r_{n} \cdot x_n) \cdot (x_{n} \cdot x_{n})^{y_{n}\dvv 2} \\[0.2cm] 
                      & = & r_{n} \cdot x_n^{2 \cdot (y_{n}\dvv 2) + 1} \\[0.2cm] 
                      & = & r_{n} \cdot x_n^{y_n} \\
                      & \stackrel{i.h.}{=} & a^{b} \quad \text{\mycheck}
                  \end{eqnarray*}

            \item $y_n \mmod 2 = 0$.

                  Then we have $y_{n} = 2 \cdot (y_n\dv 2)$ and $r_{n+1} = r_n$.
                  Therefore
                  \begin{eqnarray*}
                      &   & r_{n+1} \cdot x_{n+1}^{y_{n+1}} \\[0.2cm] 
                      & = & r_{n} \cdot (x_{n} \cdot x_{n})^{y_{n}\dvv 2} \\[0.2cm] 
                      & = & r_{n} \cdot x_n^{2 \cdot (y_{n} \dvv 2)} \\[0.2cm] 
                      & = & r_{n} \cdot x_n^{y_n} \\
                      & \stackrel{i.h.}{=} & a^{b} \quad \text{\mycheck}
                  \end{eqnarray*}
            \end{enumerate}
\end{enumerate}
This shows the validity of equation (\ref{eq:powerInv}).   If the \mytt{while} loop
terminates, we must have $y_N = 0$.  If $n=N$, then equation (\ref{eq:powerInv}) yields:
\\[0.2cm]
\hspace*{1.3cm}
$$
\begin{array}[t]{cll}
                 & r_N \cdot x_N^{y_N} = a^b \\
\Leftrightarrow  & r_N \cdot x_N^{0}  = a^b \\
\Leftrightarrow  & r_N \cdot 1       = a^b  & \mbox{because $x_N \not= 0$} \\
\Leftrightarrow  & r_N               = a^b
\end{array}
$$
\\[0.2cm]
In the step from the second line to the third line we have used the fact that, since $x_0 \not= 0$, we also
have that $x_n \not= 0$ for all $n \in \{0,1, \cdots, N\}$ and therefore $x_N^0 = 1$.  Hence we have shown that $r_N = a^b$.  The
\mytt{while} loop terminates because we have 
\\[0.2cm]
\hspace*{1.3cm}
$y_{n+1} = y_n \dv 2 < y_n$ \quad as long as \quad $y_n > 0$
\\[0.2cm]
and therefore $y_n$ must eventually become $0$.  Thus we have proven that
$\mytt{power}(a,b) =a^b$ holds. \qed


\subsection{Integer Square Root}
We continue with another example that demonstrates the method of \blue{symbolic execution}.
Figure \ref{fig:Integer-Square-Root-Iterative.ipynb} shows a function that is able to compute the
\blue{integer square root} of its argument $a$ as long as $a$ is less than $2^{64}$, i.e.~if $a$ is represented as an unsigned
number, then it can be represented with 64 bits.  In the implementation of the function \texttt{root} we have
already indexed the variables.  Note that there is no 
need to index the variable \texttt{a} since this variable is never updated.  To prove the correctness of this
program, we first 
have to establish invariants for the variables $p_n$ and $r_n$.  In order to be able to formulate the invariant
for the variable \texttt{r}, we have to understand that the function \texttt{root} computes the integer square
root of its argument $a$ bit by bit starting at the most significant bit.  Let us denote the
mathematical function that computes the integer square root of a number $a$ with \texttt{isqrt}.  If we
represent $\mathtt{isqrt}(a)$ in the binary system, we have
\\[0.2cm]
\hspace*{1.3cm}
$\ds \mathtt{isqrt}(a) = \sum\limits_{i=0}^{31} b_i \cdot 2^i$, \quad where $b_i \in \{0,1\}$.
\\[0.2cm]
In order to compute $\mathtt{isqrt}(a)$ we start with the last bit $b_{31}$.  If the square of $2^{31}$ is less
than $a$, then we must have $b_{31} = 1$.  Otherwise we have $b_{31} = 0$.  Assume now that we have already computed the
bits $b_{31}$, $b_{30}$, $\cdots$, $b_{31-n)}$.  In order to compute $b_{31-(n+1)}$ we have to check whether
\\[0.2cm]
\hspace*{1.3cm}
$\ds\left(\ds 2^{32-{n+1}} + \sum\limits_{i=31-n}^{31} b_i \cdot 2^i \right)^2 \leq a$.
\\[0.2cm]
If it is, then  $b_{31-{n+1}} = 1$, else $b_{31-(n+1)} = 0$.
With this in mind, we can state the invariants that hold when the \texttt{while} loop in line 4 is entered:
\begin{enumerate}
\item $\ds p_n = 2^{31 - n}$ \quad for $n=0,\cdots,31$ and $p_{32} = 0$.
\item $\ds r_n = \sum\limits_{i=32-n}^{31} b_i \cdot 2^i$ \quad where the $b_i$ are the bits of
      $\mathtt{isqrt}(a)$ in the binary representation.
\end{enumerate}

\begin{figure}[!h]
\centering
\begin{Verbatim}[ frame         = lines, 
                  framesep      = 0.3cm, 
                  labelposition = bottomline,
                  numbers       = left,
                  numbersep     = -0.2cm,
                  xleftmargin   = 1.3cm,
                  xrightmargin  = 1.3cm,
                  codes         = {\catcode`_=8\catcode`$=3},
                  commandchars  = \\\{\},
                ]
    \colorbox{sepia}{\green{def} \blue{root}(a):}
    \colorbox{sepia}{    r$_0$ = 0}
    \colorbox{sepia}{    p$_0$ = 2 ** 31}
    \colorbox{sepia}{    \green{while} p$_n$ > 0:}
    \colorbox{sepia}{        \green{if} a >= (r$_n$ + p$_n$) ** 2:}
    \colorbox{sepia}{            r$_{n+1}$ = r$_n$ + p$_n$}
    \colorbox{sepia}{        p$_{n+1}$ = p$_n$ // 2}
    \colorbox{sepia}{    \green{return} r$_N$}
\end{Verbatim}
\vspace*{-0.3cm}
\caption{An annotated programm to compute powers.}
\label{fig:Integer-Square-Root-Iterative.ipynb}
\end{figure} % $

We proceed to prove the first invariant by induction.  
\begin{enumerate}
\item[B.C.:] $n = 0$

             $p_0 = 2^{31} = 2^{31-0}$. 
\item[I.S.:] $n \mapsto n + 1$

   $p_{n+1} = p_n \;\mathtt{/\!/}\; 2 \;\stackrel{IV}{=}\; 2^{31-n} \;\mathtt{/\!/}\; 2 \;=\; 2^{31-n-1} \;=\; 2^{31-(n+1)}$.

   This holds as long as $n+1 \leq 32$.  Since we have $p_{31} = 2^{31-31} = 2^0 = 1$ it follows that
   \\[0.2cm]
   \hspace*{1.3cm}
   $p_{32} \;=\; p_{31} \;\mathtt{/\!/}\; 2 = 1 \;\mathtt{/\!/}\; 2 = 0$.
   \\[0.2cm]
   This shows that the body of the \texttt{while} loop is executed $32$ times.  This was to be expected, as
   each execution of this loop computes one bit of the result.
\end{enumerate}
We proceed to prove the second invariant by induction.
\begin{enumerate}
\item[B.C.:] $n = 0$

  We have $r_0 = 0$ and also $\sum\limits_{i=32-0}^{31} b_i \cdot 2^i = \sum\limits_{i=32}^{31} b_i \cdot 2^i =  0$,
  since the last sum is empty.
\item[I.S.:] $n \mapsto n+1$

  Now we need to perform a case distinction with respect to the test $a \geq (r_n + p_n)^2$.
  \begin{enumerate}
  \item $a \geq (r_n + p_n)^2$.

    In this case we have that $r_n + p_n \leq \mathtt{isqrt}(a)$ and since $p_n = 2^{31-n}$ this shows that
    \\[0.2cm]
    \hspace*{1.3cm}
    $r_n + 2^{31-n} \leq \mathtt{isqrt}(a)$.
    \\[0.2cm]
    This implies that $b_{31-n} = 1$.  Therefore we have
    \\[0.2cm]
    \hspace*{1.3cm}
    $
    \begin{array}[t]{lcl}
      r_{n+1} & = & r_n + p_n \\[0.2cm] 
              & \stackrel{IV}{=} & \sum\limits_{i=32-n}^{31} b_i \cdot 2^i + 2^{31-n} \\[0.5cm]
              & = & \sum\limits_{i=32-n}^{31} b_i \cdot 2^i + b_{32-(n+1)} \cdot 2^{32-(n+1)} \\[0.5cm]
              & = & \sum\limits_{i=32-(n+1)}^{31} b_i \cdot 2^i 
    \end{array}
    $
  \item $a < (r_n + p_n)^2$. 

    In this case we have that $r_n + p_n > \mathtt{isqrt}(a)$ and since $p_n = 2^{31-n}$ this shows that
    \\[0.2cm]
    \hspace*{1.3cm}
    $r_n + 2^{31-n} > \mathtt{isqrt}(a)$.
    \\[0.2cm]
    This implies that $b_{31-n} = 0$.  Since $b_{32-(n+1)} = b_{31-n}$ we have
    \\[0.2cm]
    \hspace*{1.3cm}
    $
    \begin{array}[t]{lcl}
      r_{n+1} & = & r_n  \\[0.2cm] 
              & \stackrel{IV}{=} & \sum\limits_{i=32-n}^{31} b_i \cdot 2^i \\[0.5cm]
              & = & \sum\limits_{i=32-n}^{31} b_i \cdot 2^i + 0 \cdot 2^{32-(n+1)} \\[0.5cm]
              & = & \sum\limits_{i=32-n}^{31} b_i \cdot 2^i + b_{32-(n+1)} \cdot 2^{32-(n+1)} \\[0.5cm]
              & = & \sum\limits_{i=32-(n+1)}^{31} b_i \cdot 2^i 
    \end{array}
    $
  \end{enumerate}
\end{enumerate}
Since the program terminates when $n=32$ we have
\\[0.2cm]
\hspace*{1.3cm}
$\ds \mathtt{root}(a) = r_N = r_{32} = \sum\limits_{i=32-32}^{31} b_i \cdot 2^i = \sum\limits_{i=0}^{31} b_i \cdot 2^i = \mathtt{isqrt}(a)$. \qed
\pagebreak

\exerciseEng
Use the method of symbolic program execution to prove the correctness of the implementation of the
\href{https://en.wikipedia.org/wiki/Euclidean_algorithm}{Euclidean algorithm} that is shown in Figure
\ref{fig:gcd.stlx} on page \pageref{fig:gcd.stlx}.  During the proof 
you should make use of the fact that for all positive natural numbers $x$ and $y$ the function $\mytt{gcd}$
that computes the greatest common divisor of $x$ and $x$ satisfies the equation
\\[0.2cm]
\hspace*{1.3cm}
$\mytt{gcd}(x, y) = \mytt{gcd}(y, x \,\mytt{\%}\, y)$.
\\[0.2cm]
Furthermore, the invariant of the \mytt{while} loop is
\\[0.2cm]
\hspace*{1.3cm}
$\mytt{ggt}(x_n, y_n) = \mytt{ggt}(a, b)$ \quad where $a := x_0$ and $b := y_0$.
\\[0.2cm]
Using this invariant you should be able to prove that $\mytt{ggt}(a, b) = \mytt{gcd}(a, b)$ for all
$a, b \in \mathbb{N}$ such that $a > 0$.  Note that in order to carry out the proof you have to distinguish
between the mathematical function \mytt{gcd} that computes the greatest common divisor and the \textsl{Python}
function \mytt{ggt} that is implemented in Figure \ref{fig:gcd.stlx}.
\eox

\begin{figure}[!ht]
\centering
\begin{Verbatim}[ frame         = lines, 
                  framesep      = 0.3cm, 
                  firstnumber   = 1,
                  labelposition = bottomline,
                  numbers       = left,
                  numbersep     = -0.2cm,
                  xleftmargin   = 0.8cm,
                  xrightmargin  = 0.8cm,
                  codes         = {\catcode`_=8\catcode`$=3},
                  commandchars  = \\\{\},
                ]
\colorbox{sepia}{    \green{def} \blue{ggt}(x$_0$, y$_0$):               }
\colorbox{sepia}{        \green{while} y$_n$ != 0:                       }
\colorbox{sepia}{            x$_{n+1}$, y$_{n+1}$ = y$_n$, x$_n$ % y$_n$ }
\colorbox{sepia}{        \green{return} x$_N$                            }
\end{Verbatim}
\vspace*{-0.3cm}
\caption{An iterative implementation of the Euclidean algorithm.}
\label{fig:gcd.stlx}
\end{figure}

\exerciseEng
Figure \ref{fig:square-iterative} shows a \textsl{Python} program implementing the function \texttt{square}.  Show that the equation
\\[0.2cm]
\hspace*{1.3cm}
$\mathtt{square}(s) = s^2$
\\[0.2cm]
holds for every $s \in \mathbb{N}$.  You should use symbolic execution to verify the claim.
\eox

\begin{figure}[!ht]
\centering
\begin{Verbatim}[ frame         = lines, 
                  framesep      = 0.3cm, 
                  firstnumber   = 1,
                  numbers       = left,
                  numbersep     = -0.2cm,
                  xleftmargin   = 0.8cm,
                  xrightmargin  = 0.8cm,
                  codes         = {\catcode`_=8\catcode`$=3},
                  commandchars  = \\\{\},
                ]
    def square(s$_0$):
        r$_0$ = 0
        u$_0$ = 1
        while s$_n$ > 0:
            r$_{n+1}$ += u$_n$
            u$_{n+1}$ += 2
            s$_{n+1}$ -= 1
        return r$_N$
\end{Verbatim}
\vspace*{-0.3cm}
\caption{The function \texttt{square} with annotations.}
\label{fig:square-iterative}
\end{figure}

\pagebreak

\section{Check Your Understanding}
\begin{enumerate}[(a)]
\item Explain the method of \blue{computational induction}.  
\item What are the three steps that have to be executed to prove the correctness of the implementation of a
      \textsl{Python} function via computational induction?
\item What are the two properties of \href{https://en.wikipedia.org/wiki/Euclidean_division}{Euclidean division} of natural numbers?
\item Are you able to prove the correctness of a recursive implementation of Euclid's algorithm?
\item What is the difference between a variable occurring in a mathematical formula and a variable occurring in
      a \textsl{Python} program?
\item How do we transform a variable occurring in a \textsl{Python} program into a mathematical variable?
\item Explain the method of \blue{symbolic execution}.
\item Are you able to prove the correctness of an iterative implementation of Euclid's algorithm?
\item When would you use computational induction and when would you choose symbolic execution instead?
\end{enumerate}

%%% Local Variables:
%%% mode: latex
%%% TeX-master: "logic"
%%% End:
